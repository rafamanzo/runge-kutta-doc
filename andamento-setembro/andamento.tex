\documentclass[a4paper,11pt]{report}
\usepackage[T1]{fontenc}
\usepackage[utf8]{inputenc}
\usepackage{lmodern}
\usepackage[brazil]{babel}

\title{Andamento da monografia até o mês de Setembro}
\author{Giancarlo Rigo\\
        Rafael Reggiani Manzo\\
        \textbf{Supervisor:} Prof. Doutor Marcel P. Jackowski}

\begin{document}

\maketitle
\tableofcontents

\begin{abstract}
O objetivo desta monografia é implementar o método de integração numérica de Runge-Kutta de forma paralela na unidade de processamento gráfico, permitindo a reconstrução de trajetórias tridimensionais em tempo real. Isto é útil em softwares de análise de imagems médicas para reconstruir fibras musculares e do tecido branco cerebral.

Para isso estamos estudando o método matemático, sua implementação na placa gráfica, gerando protótipos e testes comparativos de performance.
\end{abstract}

\chapter{Atividades realizadas}
O principal produto até o momento é um primeiro protótipo capaz de carregar imagens no formato \textit{Analyze 7.5}, gerar as fibras para esta imagem tanto utilizando a \textit{CPU} quanto a \textit{GPU}. Depois os resultados são mostrados através de uma interface utilizando as bibliotecas \textit{Glut} e \textit{OpenGL}.

Para realizar este protótipo foi necessário o estudo dos conceitos matemáticos de problema de valor inicial, o método de integração numérica de Runge-Kutta e a técnica de interpolação trilinear. Por fim, foi preciso estudar as linguagens de programação para GPU.

Estes estudos e protótipo foram documentados nos cinco capítulos da monografia já escritos.

\chapter{Atividades em andamento}
No momento está sendo feito um segundo protótipo utilizando a biblioteca \textit{VTK} com as mesmas funcionalidades do primeiro.

\chapter{Atividades futuras}
Para complementar o trabalho, faremos testes de performance entre as implementações em \textit{GPU} e \textit{CPU}, depois criar abstrações para o método na \textit{VTK} para tornar possível seu aproveitamento em outros softwares.

\end{document}
