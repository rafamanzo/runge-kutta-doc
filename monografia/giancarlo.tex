\chapter{Giancarlo Rigo}
Os maiores desafios enfrentados durante a produção deste trabalho foram aprender uma linguagem nova, utilizar o paradigma de programação concorrente que não estavamos acostumados a utilizar e relembrar (ou até mesmo aprender) a matemática que nos foi ensina nos primeiros anos do curso de Ciência da Computação.

Minhas frustrações no trabalho ocorreram pela dificuldade de compilar, executar e mesmo depurar o código em OpenCL. Horas foram gastas para encontrar erros simples, mas que pelo compilador apresentá-los de forma genérica foram difíceis de serem detectados. Outra frustração encontrada foi com a minha GPGPU, pois as GPUs da Intel funcionam apenas em Windows, um ambiente não propício a programação, o que me levou a utilizar uma GPU remota.

As matérias consideradas introdutórias do curso de Ciência Computação, como os Cálculos e Álgebra Linear, foram fundamentais para o entendimento do problema e da realização da solução apresentada no trabalho. Entretanto, as disciplinas Programação Concorrente e Computação Gráfica foram de grande importante na realização do trabalho, pois utilizamos os conceitos de paralelismo que possibilitaram os resultados obtidos e as visualizações destes resultados.

O Trabalho de Conclusão de Curso nos proporcionou uma oportunidade de utilizar os conceitos estudados no decorrer do curso, muitos deles que no momento em que cursavamos as matérias, nos pareciam sem aplicação.

Se continuasse atuando na área de estudo deste trabalho, iria aprofundar meus conhecimentos nas linguagens OpenCL e CUDA, além das especificações técnicas das GPGPUs.
