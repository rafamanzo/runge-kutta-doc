\chapter{Conclusão}
Quando compreendemos um campo vetorial como a discretização de um sistema de equações diferenciais ordinárias, por meio de interpolações trilineares, é possível adaptar o método de Runge-Kutta para reconstruir trajetórias dentro deste campo a partir de um conjunto de pontos iniciais obtendo resultados visuais dentro do esperado, como foi demonstrado no primeiro protótipo.

Quando a quantidade de pontos iniciais é grande, o tempo de resposta para solucionar o problema também passa a ser grande ao ponto de não ser mais possível realizar o processamento em tempo real na \textit{CPU}. Indo além, podemos entender que os vetores do campo vetorial em questão fornecem apenas a direção e não o sentido, o que ainda duplica a quantidade de pontos iniciais.

Porém, visto que cada ponto e cada direção deste são instâncias independentes do problema, este tempo de processamento pode ser reduzido drasticamente em um hadware com alto paralelismo como uma \textit{GPU}, que foi comprovadamente, capaz de gerar um resultado visual idêntico ao gerado em \textit{CPU} e num tempo que pode chegar a ser até 3000 vezes menor (600s contra 0.2s para 256 pontos iniciais), como observamos nos testes de performance realizados.

Comprovados a viabilidade da implementação em \textit{GPU} e seu ganho de desempenho para este método, foi elaborado um segundo protótipo utilizando a biblioteca \textit{VTK} com as mesmas funcionalidades que o primeiro. Este é útil como base para o planejamento de quais classes desta biblioteca devem ser abstraídas para o uso de \textit{GPU}.

Portanto, a adapatação do método de Runge-Kutta para reconstrução de trajetórias tridimensionais a partir de campos vetoriais pode ser processada em \textit{GPU} com resultado visual idêntico ao obtido pelo processamento em \textit{CPU}, porém com tempo de resposta muito menor.
