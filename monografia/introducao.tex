\chapter{Introdução}

\section{Motivações}
O método de integração numérica de Runge-Kutta permite a aproximação da solução de equações diferenciais ordinárias (EDOs), podendo ser generalizado para a reconstrução de trajetórias tridimensionais a partir de campos vetoriais, que será o objeto desta monografia.

Este método é computacionalmente custoso. Isto impossibilita o seu uso em aplicações em tempo real quando programada de forma convencional para CPU. Contudo, já existem implementações que conseguem atingir a performance desejada para processamento em tempo real através do uso de uma placa gráfica dedicada (\ref{lapix}) para realizar o processamento em paralelo.

Hoje o que há de mais moderno para programação de propósito geral na placa gráfica é o uso de \textit{CUDA} ou \textit{OpenCL}. Mas, ao pesquisarmos implementações do método nestas linguagens encontramos apenas um resolvedor de equações diferenciais feito em \textit{OpenCL}(\ref{CLRungeKutta46}) e nada sobre reconstrução de trajetórias tridimensionais foi feito em ambas as linguagens.

Por fim, uma das aplicações para este método é a reconstrução de fibras musculares ou do tecido branco cerebral a partir de imagens por difusão de tensores (conhecido por tractografia). Isto é uma funcionalidade comum em softwares para exploração de imagens médicas, como o \textit{MedSquare} (\ref{medsquare}).

\section{Objetivos}
Primeiramente, é preciso criar um protótipo em \textit{C++} capaz de processar o \textit{dataset} de entrada, aplicar o método e exibir seus resultados. Isto será útil para ganhar familiaridade com todos os aspectos que envolvem a implementação.

\newpage
Após, como este método é computacionalmente custoso porém altamente paralelizável, o segundo objetivo é tê-lo implementado para \textit{GPU}, com \textit{CUDA} e \textit{OpenCL}, de forma que uma aplicação seja capaz de fazer sucessivas chamadas ao algoritmo e cada uma destas responda em um tempo curto o bastante para ser considerado em tempo real.

Com estas três implementações em prontas, para ganhar familiaridade com a \textit{VTK}, será igualmente importante fazer um prótipo equivalente ao anterior, mas que use esta biblioteca que é muito poderosa e difundida, mas igualmente complicada.

Alpem destes protótipos e tão importante quanto será a criação de testes de performance para as implementações obtidas para garantir que, de fato, a implementação em \textit{GPU} é mais eficiente que a convencional em \textit{CPU} e também mensurar qual a diferença de eficiência entre elas.

Por fim, quando alcançados estes objetivos, será possível implementar este algoritmo como um \textit{pipeline} para o \textit{VTK}, permitindo a implementação no software livre \textit{MedSquare}$^{\ref{medsquare}}$ onde um de seus possíveis usos será a adição da funcionalidade de tractografia em tempo real (\textit{real time fiber tracking}$^{\ref{fiber-tracking-article}}$).

\section{Desafios}
Quando programamos para \textit{GPU} temos que ter em mente certas limitações da linguagem, como complexidade das estruturas de dados, a melhor forma de utilizar toda sua capacidade em paralelo, a forma mais eficiente de usar seus vários níveis de memória e, principalmente, como minimizar a transferência da \textit{RAM} para a \textit{GPU} e vice-versa. Caso contrário, muito provavelmente, a implementação em \textit{GPU} será mais lenta que uma para \textit{CPU}.

Ainda no contexto de programação para \textit{GPU}, um segundo desafio será a implementação em \textit{OpenCL} que é uma linguagem muito menos difundida que o \textit{CUDA} e ligeiramente diferente desta para implementar, pois permite que um mesmo código seja executado tanto em \textit{GPU} quanto \textit{CPU}.

Por fim, a implementação de um \textit{pipeline} para o \textit{VTK} torna a implementação do método ainda mais complexa pois ela deve se adaptar a sua arquitetura. Bem como, a adição de uma nova funcionalidade ao \textit{MedSquare}$^{\ref{medsquare}}$ pode exigir mais adaptações.
