\chapter{Rafael Reggiani Manzo}
Desenvolver este trabalho de conclusão de concurso foi fundamental para complementar minha formação como cientista da computação. Esta foi a primeira oportunidade de desenvolver um projeto durante todo um ano, com uma preocupação real com sua qualidade e interagindo com um docente frequentemente. Algo oposto aos trabalhos que usualmente são exigidos nas disciplinas onde eu trabalho neles durante no máximo um mês, com a preocupação de que o mínimo necessário esteja funcionando para o momento da correção e praticamente sem interagir com o professor.

Para alcançar este estado do trabalho, diversas disciplinas foram importantes e curiosamente duas delas são o Cálculo IV e a Álgebra Linear, que no momento quando cursei estas disciplinas jamais pensaria em utilizá-las. Estas duas disciplinas são toda a base matemática que foi necessária para ser possível reconstruir as trajetórias.

Igualmente importantes foram disciplinas básicas de computação como Princípios de Desenvolvimento de Algoritmos, onde pude ter o primeiro contato com conceitos como complexidade de algoritmos, estruturas de dados e até programação dinâmica.

Da mesma forma, na parte final do curso a disciplina de Computação Gráfica foi fundamental para tornar possível elaborar os protótipos que são o maior desta monografia.

Por fim, todas as disciplinas merecem o mérito por mostrarem, talvez não da melhor forma, como é possível aprender por conta própria o que é preciso para alcançar seu objetivo.
